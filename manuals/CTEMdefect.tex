
\documentclass[DIV=calc, paper=letter, fontsize=11pt]{scrartcl}	 % A4 paper and 11pt font size
% this is another comment that Mike is writing

\usepackage[body={6.5in,9.0in},
  top=1.0in, left=1.0in]{geometry}
  
\usepackage[english]{babel} % English language/hyphenation
\usepackage[protrusion=true,expansion=true]{microtype} % Better typography
\usepackage{amsmath,amsfonts,amsthm} % Math packages
\usepackage[svgnames]{xcolor} % Enabling colors by their 'svgnames'
\usepackage[hang, small,labelfont=bf,up,textfont=it,up]{caption} % Custom captions under/above floats in tables or figures
\usepackage{booktabs} % Horizontal rules in tables
\usepackage{fix-cm}	 % Custom font sizes - used for the initial letter in the document

\usepackage{sectsty} % Enables custom section titles
\allsectionsfont{\usefont{OT1}{phv}{b}{n}} % Change the font of all section commands

\usepackage{fancyhdr} % Needed to define custom headers/footers
\pagestyle{fancy} % Enables the custom headers/footers
\usepackage{lastpage} % Used to determine the number of pages in the document (for "Page X of Total")
\usepackage[squaren]{mySIunits}
\usepackage{epsfig}

\usepackage{fancyvrb}% used to include files verbatim
%\usepackage{chemsym}

\usepackage{hyperref}

\usepackage[backend=bibtex,style=numeric,sorting=ydnt,maxnames=15]{biblatex}
\renewbibmacro{in:}{}

% Count total number of entries in each refsection
\AtDataInput{%
  \csnumgdef{entrycount:\therefsection}{%
    \csuse{entrycount:\therefsection}+1}}

% Print the labelnumber as the total number of entries in the
% current refsection, minus the actual labelnumber, plus one
\DeclareFieldFormat{labelnumber}{\mkbibdesc{#1}}    
\newrobustcmd*{\mkbibdesc}[1]{%
  \number\numexpr\csuse{entrycount:\therefsection}+1-#1\relax}


%\addbibresource[label=papers]{mypubs.bib}
%\addbibresource[label=books]{mypubs.bib}
%\addbibresource[label=edited]{mypubs.bib}
%\addbibresource[label=chapters]{mypubs.bib}


% Headers - all currently empty
\lhead{}
\chead{}
\rhead{}

% Footers
\lfoot{\textsf{CTEMZAdefect/CTEMSRdefect} manual, v1.0, \today}
\cfoot{}
\rfoot{\footnotesize Page \thepage\ of \pageref{LastPage}} % "Page 1 of 2"

\renewcommand{\headrulewidth}{0.0pt} % No header rule
\renewcommand{\footrulewidth}{0.4pt} % Thin footer rule

\usepackage{lettrine} % Package to accentuate the first letter of the text
\newcommand{\initial}[1]{ % Defines the command and style for the first letter
\lettrine[lines=3,lhang=0.3,nindent=0em]{
\color{DarkGoldenrod}
{\textsf{#1}}}{}}

\usepackage{titling} % Allows custom title configuration

\newcommand{\HorRule}{\color{DarkGoldenrod} \rule{\linewidth}{1pt}} % Defines the gold horizontal rule around the title

\pretitle{\vspace{-1.5in} \begin{center} \HorRule \fontsize{25}{25} \usefont{OT1}{phv}{b}{n} \color{DarkRed} \selectfont} % Horizontal rule before the title

\title{Zone Axis and Systematic Row\\ (S)TEM Defect Image Simulations} % Your article title

\posttitle{\par\end{center}\vskip 0.5em} % Whitespace under the title

\preauthor{\begin{center}\large \lineskip 0.5em \usefont{OT1}{phv}{b}{sl} \color{DarkRed}} % Author font configuration

\author{\vspace*{-0.7in}} % Your name

\postauthor{\footnotesize \usefont{OT1}{phv}{m}{sl} \color{Black} % Configuration for the institution name

\par\end{center}\HorRule} % Horizontal rule after the title
\date{Program Manual, v1.0, \today}

\newcommand{\ctp}{\textsf{CTEMsoft-2013}}
\newcommand{\button}[1]{\colorbox{green}{\textsf{#1}} button}

\begin{document}
\maketitle


\begin{figure*}[h]
\leavevmode\centering
\epsffile{figs/CTEMlogo}
\end{figure*}

\renewcommand{\contentsname}{Table of Contents}
{\small\tableofcontents}

\newpage
\section{Introduction}
This manual describes a set of several programs, two written in Fortran-90,\footnote{f90 is a much richer language than the original fortran-f77, and is
used for all programs in the \ctp\ package.} 
the others in IDL,\footnote{The \textit{Interactive Data Language} is an interpreted scripting language with extensive graphics capabilities.} 
that can be used for the simulation of bright field (BF) and high angle annular dark field (HAADF) STEM defect images.  The f90 programs
are named \textsf{CTEMZAdefect} and \textsf{CTEMSRdefect} (note that all program in the \ctp\ package start with the letters ``CTEM'') and are capable of 
generating either computed images for a range of detector parameters, or a rather large output file which can then be visualized with the 
IDL routine \textsf{STEMDisplay.pro}.  The advantage of using a separate visualization routine is that one can freely change a large number
of parameters without having to re-calculate the underlying data.

On the following pages we will try to accomplish four tasks:
\begin{enumerate}
	\item Explain the underlying image formation theory and the numerical approach followed by the f90 program (\ref{sec:theory});
	\item Document the various input files for the f90 programs (\ref{sec:f90});
	\item Document the IDL interface (\ref{sec:idl});
	\item Explain the use of these program by means of a few basic examples (\ref{sec:examples}).
\end{enumerate}

\input{common.tex}

\newpage
\section{(S)TEM defect imaging: the underlying theory\label{sec:theory}}
The Darwin-Howie-Whelan equations of dynamical electron scattering can be formulated as a set of coupled first order 
differential equations, one for each scattered beam:
\begin{equation}
    \frac{\mathrm{d} S_{\mathbf{g}}(z)}{\mathrm{d}z} =
    2\pi\mathrm{i}s_{\mathbf{g}}S_{\mathbf{g}}(z) + \mathrm{i}\pi {\sum_{\mathbf{g}'}}
    \frac{\mathrm{e}^{-\mathrm{i}\alpha_{\mathbf{g}-\mathbf{g}'}(\mathbf{r})}}
    {q_{\mathbf{g}-\mathbf{g}'}}S_{\mathbf{g}'}(z),\label{eq:defectequation}
\end{equation}
where $\alpha_{\mathbf{g}}(\mathbf{r})=2\pi\mathbf{g}\cdot\mathbf{R}(\mathbf{r})$ and $\mathbf{R}(\mathbf{r})$ is the 
total displacement field at location $\mathbf{r}$ due to all defects in the foil.  The wave 
function of the scattered beam $\mathbf{g}$ as a function of depth $z$ in the sample is represented by:
\begin{equation}
	\psi_{\mathbf{g}}(z) = S_{\mathbf{g}}(z) \mathrm{e}^{\mathrm{i}\theta_{\mathbf{g}}},
\end{equation}
where $\theta_{\mathbf{g}}$ is the phase factor of the Fourier coefficient $U_{\mathbf{g}}$ of the
electrostatic lattice potential.  In eq.~(\ref{eq:defectequation}), $s_{\mathbf{g}}$ is the excitation
error, and the coefficients $q_{\mathbf{g}}$ are defined as:
\begin{equation}
	\frac{1}{q_{\mathbf{g}}} \equiv \frac{1}{\xi_{\mathbf{g}}} + \mathrm{i}
	\frac{\mathrm{e}^{\mathrm{i} (\theta^{\prime}_{\mathbf{g}}-\theta_{\mathbf{g}})}}{\xi^{\prime}_{\mathbf{g}}};
	\label{eq:defineq}
\end{equation}
the extinction distance $\xi_{\mathbf{g}}$ and anomalous absorption length $\xi'_{\mathbf{g}}$ are defined as:
\begin{equation}
	\frac{1}{\xi_{\mathbf{g}}}\equiv \frac{\vert U_{\mathbf{g}}\vert}{\vert\mathbf{k}_0+\mathbf{g}\vert\cos\alpha};\qquad
	\frac{1}{\xi'_{\mathbf{g}}}\equiv \frac{\vert U'_{\mathbf{g}}\vert}{\vert\mathbf{k}_0+\mathbf{g}\vert\cos\alpha};
\end{equation}
$U_{\mathbf{g}} = (2me/h^2) V_{\mathbf{g}} $, $\alpha$ is the angle between the beam direction and $\mathbf{k}_0+\mathbf{g}$, and $\mathbf{k}_0$ is 
the incident wave vector corrected for refraction.  The absorption potential Fourier coefficients are represented by $U'_{\mathbf{g}} = (2me/h^2) V'_{\mathbf{g}}$.

In matrix form, arranging the scattered amplitudes as a column vector $\mathbf{S}(z)$, one can rewrite the DHW equations as,
\begin{equation}
	\frac{\mathrm{d}\mathbf{S}(z)}{\mathrm{d}z} = \mathrm{i}\mathcal{A}(\mathbf{r})\mathbf{S}(z),\label{eq:matrix}
\end{equation}
where the structure matrix $\mathcal{A}$ contains the excitation errors and the normal absorption coefficient $1/q_{\mathbf{0}}$ 
along its diagonal, and the interaction parameters $1/q_{\mathbf{g}}$ as well as the defect phase factors on the off-diagonal positions:
\begin{align*}
    \mathcal{A}_{\mathbf{g},\mathbf{g}} & = 2\pi s_{\mathbf{g}} + \frac{\pi}{q_{\mathbf{0}}};\\
    \mathcal{A}_{\mathbf{g},\mathbf{g}'} & = \frac{\pi \mathrm{e}^{-\mathrm{i}\alpha_{\mathbf{g}-\mathbf{g}'}(\mathbf{r})}}
    {q_{\mathbf{g}-\mathbf{g}'}}
\end{align*}
The formal solution to eq.~(\ref{eq:matrix}) for a thin defect-free crystal of thickness $\epsilon$ is given by:
\begin{equation}
	\mathbf{S}(\epsilon) = e^{\mathrm{i}\mathcal{A}\epsilon}\mathbf{S}(0) = \mathcal{S}(\epsilon)\mathbf{S}(0);
\end{equation}
the matrix exponential $\mathcal{S}(z)$ is commonly known as the \textit{scattering matrix};  
the exponential of a matrix can be defined (and computed) by means of the Taylor expansion of the exponential function, i.e.,
\begin{equation}
	e^{\mathrm{i}\mathcal{A}\epsilon} = \mathbf{1}+\mathrm{i}\frac{\mathcal{A}}{1!}\epsilon-\frac{\mathcal{A}^2}{2!}\epsilon^2+\cdots+(\mathrm{i})^n\frac{\mathcal{A}^n}{n!}\epsilon^n+\cdots.\label{eq:Taylor}
\end{equation}
Since both the excitation errors $s_{\mathbf{g}}$ and the interaction coefficients $(q_{\mathbf{g}-\mathbf{g}'})^{-1}$ are small numbers, this expansion typically converges quickly for 
small values of $\epsilon$.  The solution for a foil with a thickness that is a multiple of $\epsilon$ $(z=m\epsilon)$ is then given by:
\begin{equation}
	\mathbf{S}(z) =  \mathcal{S}(\epsilon)^m\mathbf{S}(0),
\end{equation}
which is readily computed using matrix multiplications.  In the current version of the \textsf{CTEMZAdefect} and \textsf{CTEMSRdefect} programs, we employ the Pad\'e approximation 
for the efficient computation of the scattering matrix.\footnote{C. Moler and C. Van Loan, Nineteen dubious ways to compute the exponential of a matrix, 
twenty-five years later, SIAM Review 45:3-49, 2003.}  

To perform image simulations we employ the column approximation, and divide the sample into columns parallel to the incident 
beam direction; each column is subdivided into $N$ slices, so that $N\epsilon=t$, with $t$ the local foil thickness.
Once the scattering matrix is known for each slice $i$, the wave amplitudes can be computed throughout the thickness 
of the foil by successive multiplication with $\mathcal{S}_i$; for numerical efficiency, we compute the amplitude 
vectors rather than matrix products, so that the scattered amplitudes at a given layer $i$ are given by:
\[
	\mathbf{S}_i = \mathcal{S}_i\mathbf{S}_{i-1}.
\]
At the entrance surface, we have $\mathbf{S}_0=(1,0,\ldots,0)^{T}$, where the incident beam is the first entry at unit amplitude and the superscript $^{T}$ denotes the transpose.
In the absence of defects, all scattering matrices $\mathcal{S}_i$ are identical, so that the integration along a column
can be performed efficiently.  

It is important to note that the defect phase factor in equation~(\ref{eq:defectequation}) is a periodic function of its argument, so that only values of 
$\mathbf{g}\cdot\mathbf{R}\,\text{modulo}\,1$ are relevant, with $\mathbf{R}$ the total displacement vector at a given position.  If we consider a systematic row
condition, with all reflections written as integer multiples of a fundamental reflection $\mathbf{G}$, then the phase factor becomes
\begin{equation}\label{eq:phasefactor}
	e^{-\mathrm{i}\alpha_{\mathbf{g}-\mathbf{g}'}(\mathbf{r})} \rightarrow e^{-2\pi\mathrm{i}(n-n')\mathbf{G}\cdot\mathbf{R}(\mathbf{r})}.
\end{equation}
For the one axis case, we need to define two separate independent $\mathbf{g}$ vectors, which we will refer to as $\mathbf{g}_a$ and $\mathbf{g}_b$; all 
reflections in the zone can be written as integer linear combinations of these two vectors.  It is then easy to see that the defect phase factor for
general reflections $\mathbf{g}=m\mathbf{g}_a+n\mathbf{g}_b$ and $\mathbf{g}'=m'\mathbf{g}_a+n'\mathbf{g}_b$ can be written as:
\begin{equation}\label{eq:phasefactorZA}
	e^{-\mathrm{i}\alpha_{\mathbf{g}-\mathbf{g}'}(\mathbf{r})} \rightarrow e^{-2\pi\mathrm{i}(m-m')\mathbf{g}_a\cdot\mathbf{R}(\mathbf{r})} 
	e^{-2\pi\mathrm{i}(n-n')\mathbf{g}_b\cdot\mathbf{R}(\mathbf{r})}.
\end{equation}
In both cases above, the defect contribution is described by a periodic phase factor, so we can pre-compute an array of scattering matrices for 
the possible range of $\mathbf{g}\cdot\mathbf{R}$ values; in the systematic row case, we sample at $0.1^{\circ}$ intervals, resulting in the pre-computation
of $3,600$ scattering matrices.  For the zone axis case, we use a coarser sampling combined with bilinear complex interpolation, and the pre-computed
array has dimensions $181\times 181$, i.e., sampling at $2^{\circ}$ intervals. The total memory
storage required for the zone axis program is mostly determined by the storage of these arrays; for $N=31$ beams, the array of scattering matrices takes up about 
$240$ Mb of RAM, and this amount scales with $N^2$.  This array must be recomputed for each incident beam orientation since the integration columns have a slightly
different orientation, leading to changes in the excitation errors.  Using this approach for defect contrast simulations was first suggested by Th{\"o}l{\'e}n in 
1970\footnote{A.R. Th{\"o}l{\'e}n,  A {R}apid {M}ethod for {O}btaining {E}lectron {M}icroscope {C}ontrast {M}aps of {V}arious {L}attice {D}efects,
{\em Phil.\ Mag.}, 22:175--182, 1970.} as a potential solution to the
geometrical restrictions of the Head \& Humble approach but it has not been widely used since.  

\section{The \protect\textsf{CTEMZAdefect.f90} program\label{sec:f90}}

\subsection{Program overview\label{sec:f90overview}}
To implement the scattering matrix approach in a flexible way is not an easy task.  The current version of 
the \textsf{CTEMZAdefect} program is about half-way to providing complete flexibility in terms of defining the 
defect configuration and computing the image for an arbitrary incident beam orientation.  In a future version
of the code, there will be complete flexibility with implementation of the scattering matrix formalism that 
includes Bethe potentials, but in the present version the goals were a little more modest.

The program consists of three broad segments: 
\begin{itemize}
	\item initialization of foil geometry, defect geometry, beam direction, active reflections, and detector geometry;
	\item computation of the overall defect displacement array;
	\item generation of and multiplication of the appropriate sequence of scattering matrices for each image column.
\end{itemize}
Interaction with the program is predominantly by means of namelist files, which are described in the following sections.

The program can produce conventional BF/DF images for a given defect configuration and diffraction condition; in that
case, the output consists of a single file with $\mathbf{g}$-vectors and images.  Alternatively, the program can produce 
simulations for STEM mode, in which case
a much larger output file is produced; both cases can be analyzed using the \textsf{STEMDisplay} program.  The IDL interface
has the ability to convert simulated images to basic image file formats.

\subsection{Namelist input files\label{sec:f90input}}
All user input is performed via namelist files; the basic format of a namelist file is explained in the \ctp\ manual.
Below we define each of the input files needed for the \textsf{CTEMZAdefect} program.  Remember that each \ctp\ program
has a command line option to automatically generate all relevant template namelist files in the current folder.

\subsubsection{CTEMZAdefect.nml\label{sec:f90input1}}
This is the main namelist file for this program.  If the filename is different from \textsf{CTEMZAdefect.nml}, then that
new name must be passed on to the program as a command line argument; if no filename is provided, then the program expects 
a file by the name \textsf{CTEMZAdefect.nml} in the folder in which the program is executed. 

\fvset{frame=lines,formatcom=\color{blue},fontsize=\footnotesize}
\VerbatimInput{../templatefolder/CTEMZAdefect.template}


\subsubsection{STEM\_rundata.nml\label{sec:f90input2}}
The program will look for a namelist file with parameters that are related to the STEM detector and camera length setup.  The file 
template is as follows:
\fvset{frame=lines,formatcom=\color{blue},fontsize=\footnotesize}
\VerbatimInput{../templatefolder/STEM_rundata.template}
The final entry, \textsf{numberofsvalues}, refers to how many points should be used to sample the incident beam cone; the program
will take the beam convergence angle and subdivide it into \textsf{numberofsvalues} equal-size segments.  This means that the
computation time will be proportional to the square of \textsf{numberofsvalues}; the total number of incident beam directions 
inside the cone is approximately $\pi\times$ \textsf{numberofsvalues}$^2$. The larger \textsf{numberofsvalues}, the more
detail will be resolved inside the diffraction disk, but the longer the program will take.

\subsubsection{FOIL\_rundata.nml\label{sec:f90input3}}
This file defines the foil geometry, and requires a bit of explanation (see also section~\ref{sec:f90frames}).  The basic 
reference frame of the foil is defined by means of the foil normal (in direction indices) and what will become the 
horizontal image direction (defined as the reciprocal lattice vector pointing towards the airlock along the primary tilt axis).
\fvset{frame=lines,formatcom=\color{blue},fontsize=\footnotesize}
\VerbatimInput{../templatefolder/FOIL_rundata.template}
The shape of the foil can be defined as an elliptic paraboloid of the form:
\[
	z = \text{brx} \times (x-\text{cpx})^2 + \text{bry} \times (y-\text{cpy})^2 + \text{brxy} \times (x-\text{cpx})(y-\text{cpy}).
\]
$(\text{cpx},\text{cpy})$ define the center of the paraboloid; if $\text{brx}$ and $\text{bry}$ have opposite signs, then 
the foil will have a saddle surface shape, otherwise the shape will have curvature of the same sign all across.
This has not been tested extensively, but one can basically define the foil to be curved according to this
relation; foil curvature changes the local excitation error and hence produces bend contrast.  This is just
added for fun, since it is virtually impossible to determine the exact foil shape.  The prefactors in the equation
above should be kept of the order of $10^{-7}$ or so to obtain realistic results.

In a later version of the program it will be possible to define the sample tilt angles and to have the program
determine automatically which reflections need to be considered for the image computation.  In the present 
version of \textsf{CTEMZAdefect}, \textit{the foil can be tilted but the diffraction conditions remain unchanged};
this is therefore not a very realistic way of doing things.  Automatically determining which reflections contribute
to an image is not difficult, but one can imagine entering a relatively large tilt angle, say $30^{\circ}$, which
would cause problems for the \textsf{STEMDisplay} visualization program, which expects its input data to be based 
on zone axis geometry (namely the zone axis corresponding to the foil normal).  A more comprehensive implementation
will require a nearly complete rewrite of the reflection handling part of the code, and will also require conceptual
changes to the visualization routine.  Long and short of it: keep the tilt angles at zero for now.

The foil thickness is expressed in \nano\meter, and corresponds to the untilted foil; i.e., it is the thickness measured
along the foil normal.  

Finally, the elastic moduli are needed for the computation of dislocation displacement fields.  One should only
enter the non-zero elements of the $6\times 6$ elastic moduli matrix (in Voigt notation).  The program only needs
the non-zero diagonal and upper diagonal values and will apply symmetry to obtain the complete matrix.  The units 
of the moduli do not matter, as long as all of them are the same. The program only uses relative values, scaled 
to the $(1,1)$ value.



\subsubsection{Defect nml files\label{sec:f90input4}}

Dislocations can be defined by means of the following namelist template:
\fvset{frame=lines,formatcom=\color{blue},fontsize=\footnotesize}
\VerbatimInput{../templatefolder/dislocation.template}
Note that a dislocation inherently has a discontinuity plane and a logarithmic singularity at the core,
which from a mathematical point of view means that one can encounter non-physical values in the image
computation, which might give rise to artifacts located at the core of the dislocation.  To minimize the
chances of this happening, it is good practice to avoid placing the dislocation at ``simple'' positions;
instead, one should use coordinates of the type $0.501$ instead of $0.5$, and so on.  The reference frame
will be defined in the following section.

For dislocations with line directions normal to the foil normal, one must also specifiy the \textsf{zfrac}
parameter, which indicates the depth of the dislocation with respect to the foil surfaces. The depth parameter
equals $0$ at the top surface, and $1$ at the foil bottom surface.\\

\noindent For stacking faults, we need to define the fault plane, both partials, and their separation (in \nano\meter):
\fvset{frame=lines,formatcom=\color{blue},fontsize=\footnotesize}
\VerbatimInput{../templatefolder/stackingfault.template}

\noindent For voids, we use a basic text file:
\fvset{frame=lines,formatcom=\color{blue},fontsize=\footnotesize}
\VerbatimInput{../templatefolder/void.template}
Note that this is not a namelist file, so it is not possible to put comments in the file.
The first line sets the number of voids.  Each following line describes the location (3 parameters)
and radius (last number) of the void.  The void radius is defined in nanometers, the position is 
defined in the image reference frame, with the $x$ and $y$ coordinates in the interval $[-1,1]$, 
and the $z$-coordinate in the interval $[0,1]$, with $z=0$ the top of the foil and $z=1$ the bottom.

Finally, for small inclusions, we use a similar format with five columns, the final column denoting the 
value of the parameter $C$ in equation $(8.36)$ of the CTEM text book.
\fvset{frame=lines,formatcom=\color{blue},fontsize=\footnotesize}
\VerbatimInput{../templatefolder/inclusion.template}


\subsection{Program modes\label{sec:f90modes}}
The \textsf{CTEMZAdefect} program can be executed in three different modes defined by the \textsf{progmode} parameter:
\begin{itemize}
	\item \textit{CTEM mode}: When the \textsf{progmode} parameter in the main input file is set to \textsf{CTEM},
	then the program will operate with a single incident beam direction along the specified zone axis.  The algorithm will
	automatically determine which beams to take into account, based on a combination of the \textsf{dmin} parameter and the 
	setting of the \textsf{cutoff} parameters in the \textsf{BetheParameters.nml} file.  The output consists of a file with 
	a single BF image and as many DF images as there are independent reflections.  In this mode, the \textsf{STEMnmlfile} 
	namelist file is ignored.
	
	\item \textit{BF/HAADF mode}: In this mode, the \textsf{progmode} parameter in the main input file is set to \textsf{BFDF}.
	The program reads the \textsf{STEMnmlfile} and computes BF/HAADF image pairs
	for all camera lengths specified in this file.
	
	\item \textit{Full STEM mode}: This is the most general mode, which requires \textsf{progmode} to be set to \textsf{STEM}.
	A large data file is created with a CBED pattern for each image pixel (although internally,
	the data is stored in a different way).  Camera lengths in \textsf{STEMnmlfile} are ignored. 
	This mode allows for the user to define the detector parameters afterwards, via the IDL \textsf{STEMDisplay} visualization tool.
\end{itemize}

\subsection{Creating image series\label{sec:series}}
Sometimes it is useful to display simulated images as a function of some parameter, for instance the excitation error, or
the microscope voltage.  The IDL visualization program described later in this document has a operation mode in which 
such an image series is generated automatically from a series of existing data files.  From the point of view of the fortran
programs, no special steps need to be taken to accomplish this, except for the data file naming.

To enable the automatic generation of a series of images, based on a series of data files, a special file naming and numbering
convention must be followed.  Each data file in a series must have a filename that is structured as follows:
\begin{verbatim}
	fileroot__####.extension
\end{verbatim}
\textsf{fileroot} can be any string that does not contain two consecutive underscore characters; the symbol \textsf{\#\#\#\#} stands for 
a four digit sequential number, e.g., \textsf{0001}; the extension can be any string, but typically would be \textsf{data} or something similar.
Note that there must be a double underscore separating the \textsf{fileroot} string from the sequential number (this is how the visualization 
program recognizes that the file is part of a series).  The starting number in the series can be any number, but consecutive files must have
consecutive numbers, and all numbers must be formatted with four digits (including leading zeroes).  This convention works for both the 
zone axis and systematic row defect simulation programs.  Note that the amount of data generated for an image series can be very large,
so make sure you have sufficient disk space available before starting image series runs.


\subsection{Coordinate systems\label{sec:f90frames}}
To make effective use of this program, it is of utter importance to fully understand the coordinate systems that are being used
to describe the relative positions of all defects. These reference frames have been defined in detail in the CTEM text book, but there are 
some subtleties involved and a few changes have been made since the book was published in 2003. 
The basic output reference frame is the \textit{image coordinate system},
which we describe first.  The output image is a rectangular array of \textsf{DF\_npix}$\times$\textsf{DF\_npiy} pixels; each
pixel is square, with an edge length of \textsf{DF\_L} nanometers, which sets the effective magnification.  For instance, if 
the image dimensions are set to $512\times 384$, with \textsf{DF\_L}$=2.5$ \nano\meter, then the real field of view measures
$1.28\times 0.96$ \micro\squaren\meter.  The horizontal image direction will always line up with the reciprocal space direction 
$\mathbf{q}$, which is entered in the foil namelist file.  The image normal will always be the beam direction, and the vertical direction 
completes a right-handed reference frame.  

The center of the image is the origin of the defect position reference frame, and every point in the image has fractional coordinates
$(i_d,j_d)$ in the interval $[-1,+1]$.  Fig.~\ref{fig:ZAimagereferenceframe} shows schematically how the reference frame is organized.  For dislocations, 
the coordinates $(i_d,j_d)$ will always refer to the intersection of the dislocation line with the center plane of the foil.  For inclusions
and voids, the coordinates indicate the center of the defect, which involves a third coordinate to define the depth of the defect;
this coordinate varies between $0$ (foil top) and $1$ (foil bottom).  For dislocations that are parallel to the foil top and bottom 
surfaces, one must specify the depth \textsf{zfrac} in the same interval $[0,1]$.  For stacking faults, the coordinates \textsf{(SFi,SFj)} refer to the point 
midway between the partials and on the foil center plane;  once this point is defined, then the foil geometry is used along with the stacking fault
plane normal to determine the locations and surface intersection points of the partial dislocations, which may be located outside of the field 
of view for a large separation parameter \textsf{SFsep}.

\begin{figure}[h]
\centering\leavevmode
\epsffile{figs/ZAimref}
\caption{\label{fig:ZAimagereferenceframe}Definition of the image reference frame.}
\end{figure}

When the program initializes all its reference frames, it will load the crystal structure data, the foil geometry and the image dimension information first.  
From the foil geometry and the unit cell data, several reference frames and transformation matrices are defined.  The unit cell reference frame is
a non-cartesian frame (the Bravais reference frame), as is its dual, the reciprocal reference frame.  From these two, we construct a cartesian crystal
reference frame that has its $x$ direction along $\mathbf{a}$, and its $z$ direction along $\mathbf{c}^{\ast}$, with $y$ completing a right-handed triad.
All directions, including the foil normal, are defined with respect to the direct space Bravais unit cell, all reciprocal lattice vectors, the beam
direction, and the primary tilt axis (in the form of the vector $\mathbf{q}$, which is the reciprocal space vector, not necessarily with integer components,
that points along the primary tilt axis towards the sample airlock) are defined in the reciprocal reference frame; all these vectors can be 
transformed to the cartesian frame and back by a special subroutine.  The program then computes a transformation between the image cartesian frame
and the crystal cartesian frame; due to the efficiency of quaternions for the description of rotations, all transformations between orthonormal 
reference frames are carried out internally with quaternion multiplication.

Once the coordinate transformations are set up, the foil can be tilted along either of the two orthogonal primary and secondary tilt
axes, as illustrated in Fig.~\ref{fig:ZAimagereferenceframe} (note that counterclockwise rotations are positive for both $\alpha$ 
and $\beta$ tilt, looking at the origin from points located on the positive $x$ and positive $y$ image axes).  
Note that in a future version of the program, the sample tilt angles will be used to determine what the beam direction is, so that one can simply
define foil tilt angles and the program will then determine which reflections contribute and what the actual beam direction is.\footnote{It
is obviously useful to always set things up so that simple (low index) directions are parallel to the main image directions
for the zero tilt orientation, but the program will be capable in the future of dealing with more general orientations as well.}
For the actual image calculation, the foil and image reference frames are kept untilted, and it is the incident beam, in the form
of the orientation of the columns in the column approximation, that is being tilted.


\subsection{To do list\label{sec:f90todo}}

This program is a work in progress.  It was first created in March, 2010, and has undergone many modifications since.
There is an older sister program, \textsf{CTEMSRdefect}, that performs similar computations for the systematic row case, since the mathematics
is a little simpler in that case (although the weight factors for BF and HAADF detectors are more complicated).  This
is an artificial separation, though, since systematic row orientations can be reached starting from zone axis orientations
by tilting around certain directions.  The current implementation does have the ability to include HOLZ reflections when
needed (the program automatically decides which reflections are important); at the moment, Bethe potentials are not yet 
implemented, but this is just a matter of time and is a relatively high priority, with as an immediate payoff a considerable 
reduction in the time it takes to compute a full HAADF-STEM data set. 

So, here's a list of things that would be nice to have:
\begin{itemize}
	\item ability to deal with \textit{any} incident beam direction, so that the systematic row case becomes an automatic part of the program;
	\item automatic selection of relevant reflections for each incident beam direction, with inclusion of Bethe potentials;
	\item all output should be either in the form of image and data files (which has already been implemented) or in the form
	of an HDF5 data file for subsequent analysis using the \textsf{STEMDisplay.pro} routine.  HDF5 should really be used for all 
	the programs in the \ctp\ package, but that is a considerably larger task that is pushed into the future.
\end{itemize}
The first of these items will automatically happen if we can get the second one resolved.  The third item is a much longer 
term issue, but it would be really nice to get that to work.

\section{The \protect\textsf{CTEMSRdefect.f90} program\label{sec:f90SR}}
The \textsf{CTEMSRdefect} program is a separate program, but it shares most of the input files with the \textsf{CTEMZAdefect} program.  
There are only a few differences in terms of input.  The main input file with default name \textsf{CTEMSRdefect.nml} has the following entries:
\fvset{frame=lines,formatcom=\color{blue},fontsize=\footnotesize}
\VerbatimInput{../templatefolder/CTEMSRdefect.template}
Note that the parameter \textsf{progmode} has only two possible settings: \textsf{BFDF}, which generates BF/HAADF image pairs for the 
detector settings in the \textsf{STEM} name list  file, or \textsf{STEM}, which produces an output file that can be used for interactive 
detector setting adjustments.  Both types of output file can be read by the \textsf{STEMDisplay} visualization program.

To define the systematic row, two new parameters are needed: \textsf{SRG}, which represents the fundamental systematic 
row vector $\mathbf{G}$ (integer Miller indices), and \textsf{Grange}, which defines how many reflections need to be taken into
account.  The number of beams is then equal to $2$\textsf{Grange}$+1$, with the beams numbered from -\textsf{Grange*SRG} to
+\textsf{Grange*SRG}.  The \textsf{GLaue} parameter refers to the location of the Laue center along the systematic row,
measured in units of $k_t/\vert\mathbf{G}\vert$; a value of $0.5$ corresponds to the Bragg orientation being satisfied at the 
center of the $\mathbf{G}$ diffraction disk.

One should note that the \textsf{numberofsvalues} parameter in the \textsf{STEM\_rundata.nml} file has a different meaning for the 
systematic row program (for historical reasons); the value of \textsf{numberofsvalues} for the \textsf{CTEMSRdefect} program is 
equal to the number of points along the \textit{diameter} of the diffraction disk, not along the radius, as is the case for the 
zone axis program.  

All other input parameters, including the STEM and defect input files, are identical to the ones needed for the zone axis program.  
Note that at some future time, these programs
will be merged into a single program, and the algorithm will decide automatically which beams need to be taken into account.  


\section{The \protect\textsf{STEMDisplay.pro} program\label{sec:idl}}

\subsection{Program overview\label{sec:idloverview}}
When the \textsf{progmode} program parameter is set to \textsf{STEM} in the \textsf{CTEMZAdefect.nml} namelist file, the program will
create a generally rather large data file.  The data file consists of geometrical information and, for each image
pixel, the intensity of each participating reflection for each incident beam direction; this data file can become quite large in size, easily multiple gigabytes.
The main reason for storing this data is that it then becomes possible to interactively play with a series of detector parameters,
in particular the nature of the dark field detector (segmented or not).  The same program can be used to display the output 
from the other two program modes described in section~\ref{sec:f90modes}.

The \textsf{STEMDisplay.pro} routine requires an IDL program license, or one can execute the program in the Virtual Machine mode.  
For a licensed implementation, make sure that the folder containing the routines is part of your IDL pathname,
and that IDL is properly installed for the UNIX shell that you will be using (csh, bsh, etc...).
To execute the program, first start an IDL session in a terminal window (using the /Appplications/Utilities/Terminal program), 
start IDL, compile the \textsf{STEMdisplay.pro} routine by typing:
\begin{verbatim}
	IDL> .r STEMDisplay <return>
\end{verbatim}
at the IDL prompt and hitting return, followed by typing 
\begin{verbatim}
	IDL> STEMDisplay <return>
\end{verbatim}
to start the display program.  The main program widget window will appear as well as a file selector window.  
Details of all the windows are described in the following subsections.

All IDL programs, either in licensed form or via the Virtual Machine, expect the X-windows environment to be installed.  On the Mac, this corresponds
to the \textsf{XQuartz} program which can be downloaded from the open source site \textsf{http://xquartz.macosforge.org/}.  This program
must be installed in the system /Applications/Utilities folder and requires OS X 10.6 or later.

Useful thing to know: if the \textsf{STEMDisplay.pro}  program hangs for some reason (in licensed mode), you can reset your IDL session by typing
\begin{verbatim}
	IDL> .reset <return>
\end{verbatim}
This will destroy all program widgets and reset IDL to its original state.  If that does not work, then you may have to ``force quit'' the Terminal program
using the standard Esc-Option-Command key stroke.

\subsection{Main window\label{sec:idlmain}}
The main window contains most of the parameters that can be set by the user.  When the program first starts, a program preferences file 
will be read (see section~\ref{sec:idlpref}). The user can then select which type of data file to read by clicking on any of the file load buttons at the bottom.
Let's assume for now that we load a zone axis STEM file.  Use the file selection controls to navigate to the folder where your data file resides,
and select that file, either by double clicking the file name, or single clicking followed by clicking on the \button{enter}.  The program will now 
read the first section of the data file, which contains various geometrical parameters as well as a list of all potential reflections and all 
incident beam directions in the illumination cone.  Then the program reads the actual diffraction data, which may take a while; check the progress bar
that is located next to the FileSize box.\footnote{It is not unusual for the f90 program to create very large data files, so you must have sufficient 
memory in your system to be able to use this display program.  If insufficient memory is available, then the IDL session will halt with an error message.}

\begin{figure}[t]
\leavevmode\centering
\epsffile{figs/STEMwidget1}
\caption{\label{fig:STEMex1}Main widget of the \textsf{STEMDisplay} interface.}
\end{figure}

The \textsf{STEMDisplay} program can operate in two different modes which are selected by means of the \textsf{Diffraction Display Mode} switch (see 
Fig.~\ref{fig:STEMex1}): the available modes are \textit{HAADF-STEM} and \textit{regular dark field} (DF), which we will now describe in more detail.
Note that each mode is available for both the systematic row and the zone axis cases; this is indicated in the text field below the file loading progress bar.

\subsubsection{HAADF-STEM program mode}  
In HAADF-STEM mode, the display 
window will show a schematic of the HAADF (green) and BF (red) detectors along with a superimposed schematic CBED pattern with a blue disk
outline for each reflection.  The detector parameters can be set in a number of ways: the BF and HAADF detector radii can be changed (all changes
will be updated immediately in the detector schematic), as well as the camera length, which changes the size of the CBED disk outlines with respect 
to the detector.  The option \textit{\# detector segments} is somewhat experimental; the detector can be subdivided into radial segments, and the orientation
of the segment cuts can be set with the offset angle (clockwise in degrees). 

\begin{figure}[t]
\leavevmode\centering
\epsffile{figs/STEMwidget2}
\caption{\label{fig:STEMex2}BF and HAADF/DF display window.}
\end{figure}

If there is only one HAADF detector segment, and the \textit{Sector Selection Mode} is set to \textit{Single}, then pressing on the \button{Go}\ 
will create a new widget with two graphics display windows, one for the BF image, the other for the HAADF image in this mode (in the regular Dark Field
mode, this window will contain the DF image).  An example of this window is shown in Fig.~\ref{fig:STEMex2}.  After a short time, both BF and HAADF images 
will be displayed.  Using the controls in this widget, one can save the image pair in either \textit{.jpeg}, \textit{.tiff}, or \textit{.bmp} file
format.  One can also display a scale bar in the BF window; check the log window to determine what length the bar corresponds to.\footnote{It is assumed 
that the user will add the scale marker text in a separate program, such as GIMP or Illustrator, so this program only draws the scale bar without text.}
Both of the display windows indicate the minimum and maximum total intensity present in the image; each image is scaled separately so that 
black corresponds to the minimum intensity and white to the maximum.

If the \textit{Sector Selection Mode} is set to \textit{Single}, and the detector has multiple sectors, then selecting a sector by clicking on it 
in the schematic window, followed by clicking on the \button{Go}\ will display 
the HAADF image from that sector only in the HAADF/BF window.  Setting the \textit{Sector Selection Mode} to \textit{Multiple} will allow the 
user to select multiple sectors, and then display the corresponding HAADF image by clicking the \button{Go}.  This option was included as a 
test to determine whether or not multi-sector detectors can be used to perform a basic $\mathbf{g}\cdot\mathbf{b}$ analysis by selecting one or
more sectors and analyzing the resulting HAADF images.  When the \button{Save}\ is used, the program will save whatever is currently displayed in
the BF and HAADF/DF windows as a single image.

The BF window is sensitive to mouse clicks; when the user clicks on any point in that window, the program will create a 
new graphics window (an example is shown in Fig.~\ref{fig:STEMex3}) with the corresponding CBED pattern for that location.
Clicking on different points near or on the defect(s) then allows the user to display the CBED pattern, and save it using the
controls at the bottom of the window.  The CBED pattern scale can be changed with the \textsf{Zoom Factor} buttons; for a 
zoom factor of $1$, the pattern is displayed at its native magnification, meaning the each CBED disk will have as many pixels 
in it as there are individual incident beam directions (check the \textit{\# of k-vectors} entry on the main program panel).
Increasing the magnification of the pattern produces pixelated CBED patterns, since there is insufficient resolution in the 
data file to properly display all the intensity values.  The scale bar can be toggled on and off, and properly displays
the scale of the CBED pattern in \nano\meter$^{-1}$; check the program log window to find out what the actual length of the 
scale bar is.  Usually, the best patterns are obtained with the logarithmic intensity scaling mode, but this can be switched
to linear scaling as well.  As before, the CBED pattern can be stored in one of three different image file formats using the 
format selector and the \button{Save}.  Clicking on the \button{Close}\ will remove this window; it will be recreated automatically
whenever it is needed.

\begin{figure}[t]
\leavevmode\centering
\epsffile{figs/STEMwidget3}
\caption{\label{fig:STEMex3}CBED display window.}
\end{figure}


\begin{figure}[b]
\leavevmode\centering
\epsffile{figs/STEMwidget4}
\caption{\label{fig:STEMex4}Detector display window in regular dark field mode.}
\end{figure}


\subsubsection{Regular Dark Field program mode}
The second major operational mode of the \textsf{STEMDisplay} program provides the option to compute standard BF/DF
image pairs using an aperture superimposed on the CBED pattern.  When this option is selected, the detector display 
window in the main widget changes to a schematic diffraction pattern (blue disk outlines), with a cross through the 
center BF disk (see Fig.~\ref{fig:STEMex4}).  This mode is controlled by the row of options at the bottom of the
detector display window.  The aperture radius (in mrad) can be set to any value less than the beam convergence angle.
The first thing to do is to position the aperture in the BF disk by making sure the \button{set k vector} is 
selected and then clicking at the appropriate location inside the BF disk.  Note that the radius of the aperture 
will automatically be reduced to the largest possible radius if the clicked position is too close to the edge of 
the diffraction disk; the aperture must fall entirely inside the BF disk.  Once the aperture position has been selected,
the BF image will be updated to show an image obtained only with the incident wave vectors inside the aperture.
After clicking the \button{select g}, one can select any of the dark field disks inside the detector window, and the 
corresponding DF image will be displayed next to the BF image.

\subsubsection{Systematic row data files}
The \textsf{STEMDisplay} program can also display data files from the \textsf{CTEMSRdefect} program.  In this case, the 
diffraction pattern consists of only a single row of diffraction disks.  The functionality of all the program options remains 
unchanged for this type of data file, so there is no need for a separate description.  It should be noted that the input file
for the systematic row mode is typically quite a bit smaller than for the zone axis case.  This is because one only needs
to compute the scattered intensities along the central line of the systematic row, and then copy those values along lines
normal to the systematic row to reconstruct the diffraction disks.  Only the central line is stored by the \textsf{CTEMSRdefect} program,
and the display program copies the intensities to the correct locations in each of the diffraction disks.  This is an approximation,
since there are no HOLZ contributions to these patterns.  As stated before, in a later version of the \textsf{CTEMZAdefect} program,
it will be possible to define any incident beam orientation on or off zone axis; combined with sample tilt, this will be used to 
automatically determine all the relevant diffracted beams, and the systematic row case will simply become a special case that does
not require a separate program.  The current \textsf{CTEMSRdefect} program is provided as an approximate tool for 
defect image computation; the program is much faster than the zone axis version, since it only has to compute intensities along
the central line of the diffraction disks.

\subsubsection{Generation of image series}
As stated in section~\ref{sec:series}, it is possible to generate image series automatically, given a series of data files
that were generated by either of the two fortran programs (zone axis or systematic row), provided the output files adhere 
to a particular file naming convention.  When a data file with a name satisfying the naming convention is opened using the 
standard mechanism, the program will automatically recognize that this file is part of a series, and will determine how many
files there are in the series.  Then the user can proceed in the usual way, setting various parameters and/or the program modes
described in the next section.  Once the proper parameters have been set, the user can click on the \button{Series} in the 
BF/HAADF Image Display window, and each data file from the series will be loaded individually, all current imaging 
parameters will be applied to the data, and the corresponding images will be stored in image files.  The program will
repeat this operation sequentially for each file in the series.  All image files will be placed in a separate folder that has
a name that starts with the \textsf{fileroot} (see section~\ref{sec:series}) followed by a sequential number.  Repeated 
clicks on the \button{Series} will generate separate folders with images.  The user could then proceed with other programs,
such as ImageJ or Fiji, to create a movie from the individual image files.

\subsubsection{Displaying data files from other program modes}
As described in section~\ref{sec:f90modes}, the \textsf{CTEMZAdefect} program can be used in two other modes besides
the full STEM mode: CTEM, and BF/HAADF; the \textsf{CTEMSRdefect} program has a single alternative mode for BF/HAADF 
images. Such files can be loaded with the \button{CTEM File} and \button{BF/HAADF File}
at the bottom of the main window.
\begin{itemize}
\item\textbf{CTEM data files}: After selecting a data file that contains CTEM BF and DF images, a new widget will
pop up with two graphics windows, one for the BF image, and one for the DF image.  All the available reflections will
be listed, and the user can select any one of them to be displayed.  The usual controls are available to save BF/DF
image pairs.

\item\textbf{BF/HAADF data files}: After selecting a data file, a new widget appears that allows the user to 
select one of the available camera lengths and display the corresponding BF and HAADF images.  The usual controls
are available to save BF/HAADF image pairs.
\end{itemize}

\subsection{Preferences file\label{sec:idlpref}}
Upon the first execution of the \textsf{STEMDisplay.pro} routine, a preferences file will be created in the user's home folder.  The file is called \textsf{.STEMgui.prefs}; the starting period
means that the file will not show up in a Finder window or on the UNIX command line when a simple \textsf{ls} command is issued.  This is a 
regular editable text file consisting of name::value pairs.  The first line shows the number of entries in the file, and then each entry is listed on a 
separate line.  A commented version of the preferences file is as follows:
\fvset{frame=lines,formatcom=\color{blue},fontsize=\footnotesize,numbers=left}
\VerbatimInput{STEMgui-commented.prefs}
Note that the comments are not part of the actual preferences file.  All lines must be present in the file or the program will exit with an error message.
All values above, except for BFmrad, HAADFimrad, and HAADFomrad, which are computed from the actual detector dimensions and the camera length, can be altered by the user
(although there should not be any reason for manually editing this file).
The values shown above are not default values, but represent a random snapshot of the program status after it has been used for a while.

When the program starts, it will first internally initialize all variables to default values, and then read the preferences
file, if it exists.  Then the widgets will be created using the preference values.  When the program is ended normally (by
pressing the \button{QUIT}), all current values, including the widget locations, are written to the preferences
file.





\section{A few worked examples\label{sec:examples}}
In this final section, we show a few simple worked examples that illustrate how to use the combination of f90 and IDL visualization programs 
to obtain simulated CTEM and STEM defect images for both systematic row and zone axis cases.\footnote{Note that all the input
files for the examples described in this section can be found in the \textsf{examples} folder; they are labeled by section number.}  

\subsection{BF/DF TEM images of a pair of dislocations\label{sec:ex1}}
We begin this series of examples with a pair of simple perfect edge and screw dislocations in fcc copper. Note that it is not very 
likely that one would carry out dislocation observations in CTEM zone axis mode, but the program is capable of performing the 
corresponding image simulations.
  
The \textsf{Ex6.1.nml} input file contains the following entries:
\fvset{frame=lines,formatcom=\color{blue},fontsize=\footnotesize}
\VerbatimInput{../examples/DefectSimulations/Ex6.1.nml}
We note that this is a regular CTEM simulation for the $[100]$ zone axis at $200$ kV, on a $256\times 256$ pixel image, with pixel size $1.0$ \nano\meter,
and the slice thickness is set to $1.0$ \nano\meter.  The input crystal structure file is \textsf{Cu.xtal} in the main \textsf{xtal} folder,\footnote{Recall that
on a UNIX system, the symbol $../$ means to go one directory up in the file system.} and we
assume that the program is executed from within the \textsf{examples} folder.

The next file to set up is the foil description file, \textsf{EX6.1\_FOIL.nml}:
\fvset{frame=lines,formatcom=\color{blue},fontsize=\footnotesize}
\VerbatimInput{../examples/DefectSimulations/Ex6.1_FOIL.nml}
We take the foil normal to be parallel to the incident beam direction, and the vector $\mathbf{q}$ that points towards the 
airlock (and is the horizontal image direction) along $\mathbf{g}_{001}$.  The foil thickness is $100$ \nano\meter, and the 
elastic moduli for copper are listed in GPa.  Note that all other parameters that are allowed in this file can be omitted,
since the default values specified in the corresponding template file are acceptable.

Finally we need to create the defect files.  We have two perfect dislocations, one edge, the other screw, and we'll take
two different line directions as well.  The namelist files are as follows:
\fvset{frame=lines,formatcom=\color{blue},fontsize=\footnotesize}
\VerbatimInput{../examples/DefectSimulations/Ex6.1.disl1.nml}
and
\fvset{frame=lines,formatcom=\color{blue},fontsize=\footnotesize}
\VerbatimInput{../examples/DefectSimulations/Ex6.1.disl2.nml}
The first dislocation is a perfect edge with $\mathbf{u}=[101]$ and $\mathbf{b}=\frac{1}{2}[10\bar{1}]$; the second one a 
perfect screw with $\mathbf{u}=[011]$ and $\mathbf{b}=\frac{1}{2}[011]$.  They are located at two different positions, 
$(0.251,0.251)$ and $(-0.251,-0.251)$; recall that these points are the locations of the intersection of the dislocation line
with the center of the foil.

The total number of beams in the simulation is partially determined by the \textsf{dmin} entry in the main input file,
but also by the settings in the \textsf{BetheParameters.nml} file:
\fvset{frame=lines,formatcom=\color{blue},fontsize=\footnotesize}
\VerbatimInput{../examples/DefectSimulations/BetheParameters.nml}
A main cutoff parameter of $70$ reduces the total number of beam to $25$.  The current implementation of the \textsf{CTEMZAdefect} program
does not implement the Bethe potentials, so the \textsf{weakcutoff} parameter is not used.

When the program was executed with the command line \textsf{../../exe/CTEMZAdefect Ex6.1.nml} on a Mac Pro ($2\times 3.06$ GHz, $6$-core
Intel Xeon with $96$ Gb RAM), a total of $37$ diffracted beams were determined to contribute to the scattering process.  
The program took about $10$ seconds to create the total displacement field (note that this generates an array
of dimensions $2\times 256\times 256\times 100$); another $17$ seconds were needed to compute the scattering matrices and
perform the column integrations, using $6$ threads.  The total memory required for the computation was just under $600$ Mb of RAM; this is mostly
taken up by the defect displacement array and the precomputed scattering matrix array. 

The output file has a size of just under $10$ Mb, and contains the BF image and $36$ DF images.  
This file must be read by the \textsf{CTEMDisplay} visualization program using the \textsf{CTEM File} option.
Note once again that this is not a very realistic case, since one would typically not perform conventional dark field imaging 
in zone axis orientation.  An example of the simulated images is shown in Fig.~\ref{fig:example61}, which
shows the bright field image and two dark field images, for the $(0\bar{2}0)$ and $(020)$ reflections.

\begin{figure}[t]
\leavevmode\centering
\epsfxsize=4in\epsffile{figs/example61}
\caption{\label{fig:example61}Example BF and DF images for test case 6.1; the DF images are for the $(0\bar{2}0)$ and $(020)$ reflections.}
\end{figure}



A more realistic simulation would make use of the main input file \textsf{Ex6.1b.nml}:
\fvset{frame=lines,formatcom=\color{blue},fontsize=\footnotesize}
\VerbatimInput{../examples/DefectSimulations/Ex6.1b.nml}
Note that all entries are identical to the previous file, except for the \textsf{progmode} parameter, which is
now set to \textsf{BFDF}, the \textsf{STEMnmlfile}, which is needed to define the camera lengths and such, and 
the name of the output file.  The \textsf{STEMnmlfile} file is as follows:
\fvset{frame=lines,formatcom=\color{blue},fontsize=\footnotesize}
\VerbatimInput{../examples/DefectSimulations/Ex6.1_STEM.nml}
Executing the command \textsf{../../exe/CTEMZAdefect Ex6.1b.nml} resulted in a program run of just about one hour, 
for a total of $81$ incident beam directions in the illumination cone.  
The number of scattered beams in this case was $37$, which is larger than the $25$ used for the CTEM simulation;
this difference is due to the fact that the STEM computation takes HOLZ contributions into account as well.
The resulting images are shown in Fig.~\ref{fig:example61b} for the indicated camera lengths.


\begin{figure}[t]
\leavevmode\centering
\epsfxsize=4in\epsffile{figs/example61b.eps}
\caption{\label{fig:example61b}BF and HAADF images for test case 6.1b as a function of camera length.}
\end{figure}

The final example in this subsection uses input file \textsf{Ex6.1c.nml}:
\fvset{frame=lines,formatcom=\color{blue},fontsize=\footnotesize}
\VerbatimInput{../examples/DefectSimulations/Ex6.1c.nml}
In this case \textsf{progmode} is set to \textsf{STEM}, so that the output is a large file that can be read using the 
\button{STEM File} in the display program.  The user can now interactively change all parameters and generate BF/HAADF
image pairs for any desired detector configuration.



\subsection{BF/HAADF systematic row STEM images of a stacking fault\label{sec:ex2}}
In this second example we will illustrate how to compute systematic row BF/HAADF images for a stacking fault in 
the $\gamma$ phase of a Ni-based superalloy; we will employ the imaging parameters used for Figures 5 and 6 in the following paper:
\begin{verbatim}
P.J. Phillips, M. Mills, and M. De Graef. 
�Systematic row and zone axis STEM defect image simulations�. 
Philosophical Magazine A 91 (2011), pp. 2081-2101.
\end{verbatim}
The input files can again be found in the \textsf{examples} folder, and are as follows:\\
\textsf{Ex6.2.nml}
\fvset{frame=lines,formatcom=\color{blue},fontsize=\footnotesize}
\VerbatimInput{../examples/DefectSimulations/Ex6.2.nml}
the foil description file \textsf{Ex6.2\_FOIL.nml}
\fvset{frame=lines,formatcom=\color{blue},fontsize=\footnotesize}
\VerbatimInput{../examples/DefectSimulations/Ex6.2_FOIL.nml}
the STEM setup file \textsf{Ex6.2\_STEM.nml}
\fvset{frame=lines,formatcom=\color{blue},fontsize=\footnotesize}
\VerbatimInput{../examples/DefectSimulations/Ex6.2_STEM.nml}
and the stacking fault descriptor file \textsf{Ex6.2.sf.nml}
\fvset{frame=lines,formatcom=\color{blue},fontsize=\footnotesize}
\VerbatimInput{../examples/DefectSimulations/Ex6.2.sf.nml}
Note that the stacking fault lies on the $(\bar{1}11)$ plane, has
leading partial burgers vector $\frac{1}{3}[1\bar{1}\bar{1}]$,
trailing partial burgers vector $\frac{1}{3}[\bar{1}11]$, and 
both dislocations have line directions $[110]$.  The separation between
the partial dislocations is set to a large number, so that the partials
themselves will not appear in the image.  Note also the rotation angle 
\textsf{foilalR}\footnote{This is the secondary rotation angle for a tilt-rotation holder.} 
which is set to $135^{\circ}$ in the \textsf{Ex6.2\_FOIL.nml}
file; this will make sure that the stacking fault appears as horizontal lines
in the images.  Omitting this rotation angle will give rise to images with 
the stacking fault fringes running at $45^{\circ}$ from bottom left to top right.

Finally, the crystal structure file \textsf{gam.xtal} is created with the 
following command (again from within the \textsf{examples/DefectSimulations} folder): 
\begin{verbatim}
	../../exe/CTEMmkxtal <Ex6.2.gam.txt
\end{verbatim}
which sends the contents of the \textsf{Ex6.2.gam.txt} file to the \textsf{CTEMmkxtal} program.
The text file contains all the commands that one would have to enter manually; creating the 
structure file in this way will minimize typographical errors in the coordinates, since one 
can simply correct them and then run the \textsf{CTEMmkxtal} program again, without having to 
retype all the coordinates.

Executing the following command
\begin{verbatim}
	../../exe/CTEMSRdefect Ex6.2.nml
\end{verbatim}
will then carry out the image simulation, which takes only about a minute using $6$ threads.  The output
file can then be opened with the \button{STEM File} in the \textsf{STEMDisplay} program and the BF/HAADF
images can be compared with Figures 5 and 6 in the above cited paper.  One
should find that the computed images agree perfectly with the ones shown in figure 5.  To obtain
the results for Figure 6, one can either replace the parameter \textsf{SRG} by $1,1,-1$, or, one can 
keep \textsf{SRG} set to $-1,-1,1$ and set \textsf{GLaue} to $-0.5$, which brings the reflection $1,1,-1$
into Bragg orientation.



\end{document}



