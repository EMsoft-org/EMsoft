\documentclass[11pt]{amsart}
\usepackage{geometry}                % See geometry.pdf to learn the layout options. There are lots.
\geometry{letterpaper}                   % ... or a4paper or a5paper or ... 
%\geometry{landscape}                % Activate for for rotated page geometry
%\usepackage[parfill]{parskip}    % Activate to begin paragraphs with an empty line rather than an indent
\usepackage{graphicx}
\usepackage{amssymb}
\usepackage{epstopdf}
\DeclareGraphicsRule{.tif}{png}{.png}{`convert #1 `dirname #1`/`basename #1 .tif`.png}

\textwidth=6.5in
\textheight=9.0in
\oddsidemargin=0.0in
\evensidemargin=0.0in
\topmargin=-0.5in

\title{Building DREAM.3D with EMsoft support on Mac OS X}
\author{Marc De Graef}
\date{\today}                                           % Activate to display a given date or no date

\newcommand{\ems}{\textsf{EMsoft}}
\newcommand{\emsb}{\textsf{Workspace/EMsoftBuild}}
\newcommand{\emsdk}{\textsf{EMsoft\_SDK}}
\newcommand{\dtd}{\textsf{DREAM.3D}}
\newcommand{\dtdsdk}{\textsf{DREAM3D\_SDK}}
\newcommand{\emtb}{\textsf{EMsoftToolbox}}
\newcommand{\ws}{\textsf{Workspace}}
\newcommand{\wsesh}{\textsf{Workspace/EMsoft/Support/SDK\_Build\_Scripts}}

\renewcommand{\baselinestretch}{1.3}

\begin{document}
\maketitle
The \emtb\ contains a number of \dtd\ filters to create \ems\ input files, or to actually
run \ems\ programs, either via system calls or via shared dynamical library calls.  To set up the development
environment for the \emtb\ requires a number of things to be in place.  This document describes which steps to take and in 
what order.  It is assumed that you start from scratch; in all cases you must pay attention to proper upper and lower case!
You must have XCode, gcc, and gfortran installed; instructions for other (non-Mac) platforms may follow in  later version
of this document.

\begin{enumerate}
\item \textit{Download the source codes}:
\begin{enumerate}
\item Create a folder called \ws\ somewhere in your home folder\footnote{In this document, pathnames will be relative to 
this folder, unless stated otherwise.}; inside this folder, create a \textsf{DREAM3D\_Plugins} folder.
\item Pull the latest source codes for \ems\ and \dtd\ into the \ws\ folder; this should create an \ems\ folder and a \textsf{DREAM3D} folder.
\item Pull the latest source code for \emtb\ into the \textsf{DREAM3D\_Plugins} folder; this should create an \textsf{EMsoftToolbox} folder.
\end{enumerate}
\item \textit{Build \ems}:
\begin{enumerate}
\item Navigate to the \wsesh\textsf{/OSX\_Build\_Scripts} folder, edit the \textsf{SDK\_Configuration.conf} file to define 
to location of the SDK as well as the location of the fortran compiler, and build the \ems\ Software Developer Kit (SDK) by executing the following command:
{\small\begin{verbatim}prompt> sudo ./Build_SDK.sh
\end{verbatim}}
Note that you will need to have sudo access.  The scripts are currently set up so that the \emsdk\ is built in the 
\textsf{/opt/EMsoft\_SDK} folder.  This can be changed by editing the \textsf{SDK\_Configuration.conf} file in the \wsesh\/ \textsf{/OSX\_Build\_Scripts} folder.  
Another logical location would be the \textsf{/Users/Shared} folder.

The build process will download an archive with several libraries from the BlueQuartz web site and unpack the archive; then each package will be 
built and installed within the SDK folder.  This takes several minutes during which the following packages will be installed:
\begin{itemize}
\item CMake: this is the preferred build environment for this entire package;
\item fftw3: fast Fourier transform package;
\item fortrancl: limited fortran-90 interface to the OpenCL language;
\item json-fortran: fortran-90 implementation of the Java Script Object Notation (json);
\item HDF5: Hierarchical Data Format libraries; these will be built with fortran support turned on.  Ignore any compilation warnings during 
the build of this library.
\end{itemize}
\item Add the following command to your \textit{.bash\_rc} file and restart your shell:
{\small
\begin{verbatim}
export PATH=$PATH:/opt/EMSoft_SDK/cmake-3.4.1-Darwin-x86_64/CMake.app/Contents/bin/
\end{verbatim}}
Note that the version numbers (3.4.1) may change over time; use the correct numbers and the correct SDK location for your particular installation. If you use
a different shell, e.g., \textit{csh}, then use the appropriate command to extend your path.  \textit{Make sure that there is no other CMake
installation anywhere in your path!}  Don't forget to restart your shell.
\item Navigate to the \textsf{Workspace} folder and create a \textsf{EMsoftBuild} subfolder; cd to the \textsf{EMsoftBuild} folder and execute the 
following command:
{\small
\begin{verbatim}
cmake -DEMsoft_SDK=/opt/EMsoft_SDK -DCMAKE_BUILD_TYPE=Debug ../EMsoft
\end{verbatim}}
This step will read the \textsf{/opt/EMsoft\_SDK/EMsoft\_SDK.cmake} file and initialize the proper variables to enable compilation of the project.
Note that this step only needs to be carried out once after each update of the SDK.
\item In the \textsf{Build} folder, execute the command
{\small
\begin{verbatim}
make -j
\end{verbatim}}
to carry out a parallel compile of the \ems\ project.
\end{enumerate}

\item \textit{Build the \dtd\ SDK:}
\begin{enumerate}
\item Navigate to the \textsf{/Users/Shared} folder and create the \textsf{DREAM3D\_SDK} folder.
\item Go to URL \textsf{http://download.qt.io/official\_releases/qt/5.5/5.5.1/} and download the 
following file:
{\small\begin{verbatim}
qt-opensource-mac-x64-clang-5.5.1.dmg
\end{verbatim}}
Once the download is completed, double click on the .dmg icon and then on the .app file to start the installation process; \textit{make sure
you set the installation folder to the following string:} \textsf{/Users/Shared/DREAM3D\_SDK/Qt5.5.1}.  Click on Continue and wait
until the installation process is completed.
\item Navigate to the \textsf{Workspace/DREAM3D/Support/Scripts/OSX\_Build\_Scripts} folder and execute the command:
{\small
\begin{verbatim}
prompt> sudo ./Build_SDK.sh
\end{verbatim}}
This will download and unpack an archive from the BlueQuartz web site; then, the script will build and install several packages:
\begin{itemize}
\item CMake
\item boost
\item Doxygen
\item DREAM3D\_Data
\item Eigen
\item HDF5
\item InsightToolkit
\item ITK
\item protobuf
\item qwt
\item threading building blocks (tbb)
\end{itemize}
The build process takes a while, and you will likely see many warnings during the compilation of HDF5 which you can safely ignore.  Note that the 
HDF5 library is built twice, once in Debug mode and once in Release mode.  In addition, this build does not enable fortran support, so it is different
from the HDF5 build in the \emsdk; both HDF5 builds are needed to enable the integration of \ems\ and \dtd.\footnote{This is due to an issue with 
the compilation of HDF5 on Mac OS X when fortran support is enabled; in this case, there is a bug that prevents the creation of dynamical libraries,
which is what \dtd\ needs for its compilation.  Hence, \ems\ uses static HDF5 libraries, which are not compatible with \dtd.  This issue only occurs on 
Mac OS X and may be resolved by the HDF-Group in the future.}
\end{enumerate}

\item \textit{Install \ems\ in the \dtdsdk\ folder:}
\begin{enumerate}
\item To enable compilation of \dtd\ with support for \ems\ routines, a release version of the \ems\ libraries needs to be installed inside the \dtdsdk.  
In a Terminal window, navigate to the \emsb\ folder and execute the following command (all on one line):
{\small\begin{verbatim}
prompt> cmake -DCMAKE_BUILD_TYPE=Release 
         -DCMAKE_INSTALL_PREFIX=/Users/Shared/DREAM3D_SDK/EMsoft ../EMsoft
\end{verbatim}}
Then recompile \ems\ with the \textsf{make -j} command, followed by \textsf{make install}.  This will install an \ems\ folder inside the \dtdsdk.
\item To return \ems\ to debug mode for further development, execute
{\small\begin{verbatim}
prompt> cmake -DCMAKE_BUILD_TYPE=Debug ../
\end{verbatim}}
followed by \textsf{make -j} to recompile the package.
\end{enumerate}

\item \textit{Tell \dtd\ where to look for the \ems\ installation:}  
\begin{enumerate}
\item Navigate to the \textsf{Workspace/DREAM3D\_Plugins/EMsoftToolbox/Support/Scripts} folder and execute the following command:
{\small\begin{verbatim}
prompt> ./Create_EMsoftToolbox_SDK.sh  /Users/Shared/DREAM3D_SDK /opt/EMsoft_SDK
\end{verbatim}}
This will create an extra CMake configuration file inside the \dtdsdk\ folder which will inform \dtd\ about the whereabouts of the \ems\ libraries.
\end{enumerate}

\item \textit{Build \dtd\ with \ems\ support:}
\begin{enumerate}
\item In a Terminal window, go to the \ws\ folder and enter the following commands:
{\small\begin{verbatim}
prompt> mkdir DREAM3D-Build-Debug
\end{verbatim}}
{\small\begin{verbatim}
prompt> open /Users/Shared/DREAM3D_SDK/cmake-3.4.1-Darwin-x86_64
\end{verbatim}}
This will open a Finder window; double click on the \textit{CMake.app} which will pop up an interactive display.  Browse to the \dtd\ source folder
and select it (should be \textsf{/path/Workspace/DREAM3D} where \textsf{path} will depend on your home folder setup); then browse to
the binaries folder that you just created and select it (\textsf{/path/Workspace/ DREAM3D-Build-Debug}).
\item Click on the \textsf{Add Entry} button and enter \textsf{DREAM3D\_SDK} for the first prompt, \textsf{Path} for the second, and \textsf{/Users/Shared/DREAM3D\_SDK}
for the third.  Accept these entries, which will pop up a line with a red background in the main Name-Value window.
\item Click once on the \textsf{Configure} button below the main window.  Many additional entries will appear with a red background.  Next,
look for the \textsf{DREAM3D\_EXTRA\_PLUGINS} entry and enter \textsf{EMsoftToolbox} (note upper and lower case!).  Then hit the \textsf{Configure}
button again until no more items with a red background appear.
\item Click on the \textsf{Generate} button; select \textit{Unix Makefiles} from the list of options and  \textit{Use default native compilers}.  
Hit Done to start the generation process.  Then quit the CMake app.
\item In the \textsf{DREAM3D-Build-Debug} folder, execute the command \textsf{make -j} to build \dtd; this will take quite a while, but should result in 
a \dtd\ executable in the \textsf{Bin} subfolder that includes the \emtb\ filters.
\end{enumerate}

\end{enumerate}

Further comments: As of February 2016, the source code structure of the \dtd\ package has been modified substantially. The source code is 
now split over several code repositories.  In principle, only the \dtd\ code needs to be pulled from the GitHub repository, and the CMake 
scripts will take care of checking out code from the other repositories, but this is only done for the initial installation.  The user needs 
to make sure that all source code folders are regularly pulled from the respective repositories (\dtd, \textsf{SIMPL}, \textsf{SIMPLView} and \textsf{CMP}).
Once \dtd\ has been installed, all other operations described in this manual are still valid.

\end{document}  
